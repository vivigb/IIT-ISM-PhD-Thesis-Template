\chapter{Your first chapter title} \label{Chapter: 1}

\newpage
%\thispagestyle{empty}
% \section{Introduction}

\justifying
% \vspace{5cm}
{\huge This is PhD thesis overleaf template for \\ IIT(ISM),Dhanbad, India. Please note that the content in this template doesn't make any sense. This is solely for the reproducible purpose}

The field of \textit{Geophysics}, specifically \textbf{Seismic Exploration}, has seen significant advancements in recent decades. This chapter provides an introduction to the topic, highlighting the importance of seismic methods in understanding Earth's subsurface. 

The key objectives of this research are as follows:
\begin{itemize}
    \item \textbf{To investigate seismic wave propagation} and its implications for subsurface imaging.
    \item To develop advanced \textit{data processing techniques} that enhance resolution.
    \item To correlate seismic attributes with geological structures.
\end{itemize}

\noindent These objectives serve as the foundation for the methodologies and results discussed in subsequent chapters.

\section{Background and Motivation}
Seismic exploration is a critical tool for studying subsurface features. It plays a vital role in fields such as:
\begin{itemize}
    \item \textbf{Resource exploration}: including oil, gas, and minerals.
    \item \textcolor{blue}{\underline{Hazard assessment}}: identifying active faults and regions prone to seismic activity.
    \item \textit{Environmental monitoring}: evaluating changes in subsurface conditions due to human activities.
\end{itemize}

The increasing demand for energy resources and disaster mitigation strategies underscores the need for accurate and efficient seismic methods.

\section{Significance of the Study}
The significance of this research lies in its ability to address the following:
\begin{enumerate}
    \item The limitations of existing seismic models in \textbf{\textit{heterogeneous terrains}}.
    \item The need for \textcolor{red}{real-time data analysis} for earthquake early warning systems.
    \item The integration of multidisciplinary approaches for a holistic understanding of subsurface dynamics.
\end{enumerate}

\section{Research Questions and Hypotheses}
This study is guided by the following research questions:
\begin{enumerate}
    \item How do variations in seismic wave velocity correlate with subsurface structures?
    \item What are the effects of \underline{anisotropy} on seismic imaging?
    \item Can machine learning algorithms improve event detection accuracy in noisy datasets?
\end{enumerate}



\section*{Example Citation}
This demonstrates how to cite a reference from the \textit{.bib} file stored in Overleaf. References can be cited in two formats: within parentheses \parencite{babu_stress_2024} or as part of the sentence like this \textcite{babu_stress_2024}. As noted in previous studies, advancements in seismic imaging are essential for understanding complex geological formations.\par

\begin{figure}
    \centering
    \includegraphics[width=1\linewidth]{Figures//C1/f_1_1.png}
    \caption{A random phD Meme.}
    \label{fig: 1.1}
\end{figure}

This figure can be referred in text as \ref{fig: 1.1}. Similarly the table canbe referred as \ref{tab: 1.1}

% Adding a Table
\begin{table}[h!]
    \centering
    \begin{tabular}{|c|c|c|}
        \hline
        \textbf{Parameter} & \textbf{Value} & \textbf{Units} \\
        \hline
        Wave Velocity & 5.0 & km/s \\
        Frequency & 10 & Hz \\
        Depth & 20 & km \\
        \hline
    \end{tabular}
    \caption{Example parameters used in seismic analysis.}
    \label{tab: 1.1}
\end{table}

% Adding an Equation
\section*{Example Equation}
The wave equation is expressed as:
\begin{equation}
    \nabla^2 u - \frac{1}{v^2} \frac{\partial^2 u}{\partial t^2} = 0
\end{equation}
where \( u \) is the displacement, \( v \) is the velocity, and \( t \) is time.


\section{Thesis outline}

\subsection{Chapter 1: Introduction}
This is an introductory chapter which states the research problem along with the listed objectives and the background study. In this chapter, the study region and relevant geological and seismotectonic settings of the respective region are also given with proper figures and maps.

\subsection{Chapter 2: The spatio-temporal analysis of seismicity in the NW Himalayan region}
Explains the seismicity and its spatio-temporal characteristics in the North West Himalayas. This chapter also explains the data types which are used for the seismicity-related studies in this thesis. This chapter mainly focuses on giving the reader a clear idea about the basic relationships between earthquake parameters which could help to define the spatio-temporal characteristics of the study region.

\subsection{Chapter 3: Stress Regimes in The NW Himalaya: Stress Field Implications Based on Focal Mechanism Solution Data}
The chapter gives a walkthrough regarding the stress regime estimation in seismically active regions in the Himalayan region using the Focal Mechanism Solution (FMS) data. The estimation of stress regime through stress tensor inversion and its seismotectonic implications in the respective regions.


\subsection{Chapter 4: Rayleigh Wave Dispersion Curve Analysis in NW Himalayan Region}
The estimation of surface wave dispersion curves in the major seismically active zones in the NW Himalayan study region and its characterization. The delineation of 2D subsurface velocity structure using the inversion of estimated dispersion curves in NW Himalaya especially in the Kinnaur region is further correlated with the previous chapter results.	

\subsection{Chapter 5: Summary and Conclusion}
Give an outline and summarize the research based on the results from previous chapters, and try to provide a feasible interpretation based on the findings. This chapter also discusses the future scope and possible extension of this research work.