\chapter{Your Second chapter Title} \label{Chapter: 2}
% Define acronyms inline
% \newacronym{mht}{MHT}{Main Himalayan Thrust}
% \newacronym{igp}{IGP}{Indo-Gangetic Plains}

\newpage
%\thispagestyle{empty}
\section{Introduction}
\justifying
{\huge A}s global climate patterns continue to evolve, understanding regional variations in precipitation and temperature has become increasingly important. The shift in weather systems, influenced by both natural and anthropogenic factors, has led to significant changes in local climates across different regions. This ongoing transformation is particularly evident in the subtropical zones, where changes in rainfall distribution and temperature extremes are altering ecosystems. The adaptation strategies employed by these ecosystems to cope with such fluctuations are crucial for assessing long-term sustainability. Furthermore, the increased frequency of extreme weather events highlights the urgent need for comprehensive climate monitoring and accurate predictive models to inform policy decisions and mitigate the adverse effects of climate change.


As global climate patterns continue to evolve, understanding regional variations in precipitation and temperature has become increasingly important. The shift in weather systems, influenced by both natural and anthropogenic factors, has led to significant changes in local climates across different regions. Climate change, driven by human activities such as deforestation, industrialization, and the burning of fossil fuels, has resulted in a steady increase in greenhouse gases in the atmosphere. These gases, particularly carbon dioxide (\(CO_2\)), methane (\(CH_4\)), and nitrous oxide (\(N_2O\)), trap heat and contribute to the warming of the Earth’s surface, a phenomenon commonly referred to as global warming. However, while global warming affects the planet as a whole, its impacts are far from uniform, and different regions experience these changes in distinct ways.

This ongoing transformation is particularly evident in the subtropical zones, where changes in rainfall distribution and temperature extremes are altering ecosystems. In many parts of the world, these regions have seen an increase in average temperatures, which has contributed to more frequent heatwaves and prolonged periods of drought. Conversely, some areas have experienced an increase in rainfall, leading to flooding and the disruption of local water cycles. These shifts in regional climates are not just causing discomfort but are also threatening the biodiversity and health of ecosystems. Species that are adapted to specific climate conditions are struggling to survive as the temperatures rise and the distribution of precipitation becomes more erratic.

The impacts of these changes extend far beyond the natural environment. The adaptation strategies employed by ecosystems to cope with such fluctuations are crucial for assessing long-term sustainability. For instance, some plant and animal species are relocating to more favorable conditions, while others are undergoing evolutionary changes to cope with altered climates. However, many species are unable to adapt quickly enough to the rapidly changing environment, leading to the risk of extinction for those that are most vulnerable. The disruption of ecosystems has ripple effects throughout the food chain, affecting everything from the availability of food for humans to the health of local wildlife populations.

In addition to the ecological challenges, the increased frequency of extreme weather events highlights the urgent need for comprehensive climate monitoring and accurate predictive models to inform policy decisions. Severe weather events such as floods, droughts, and storms are becoming more common and more intense in many parts of the world. The increased unpredictability of these events makes it harder for communities to plan and prepare, especially in regions that lack the infrastructure to handle such disasters. In many low-lying coastal areas, rising sea levels have already led to the loss of land, and storm surges have caused widespread flooding. In other areas, prolonged droughts are threatening water supplies and agricultural production, making it difficult to sustain livelihoods.

The risk of these extreme events is further compounded by the continued increase in global temperatures. Research has shown that as the planet warms, the intensity of storms and hurricanes is likely to rise, and areas that have already been prone to flooding or droughts may experience even more extreme conditions. For instance, in the tropics, warmer sea surface temperatures can fuel more powerful hurricanes, leading to more devastating storms. Similarly, in the Arctic, the melting of ice caps is causing sea levels to rise, which could eventually lead to the displacement of millions of people living in coastal areas.

These changes in regional weather patterns also have profound implications for agriculture, water resources, and public health. In many parts of the world, farmers are already facing the brunt of changing weather conditions, with altered growing seasons, unpredictable rainfall, and increased pest infestations. Crops that once thrived in certain areas are now struggling, and new agricultural practices are required to adapt to the changing conditions. In regions that rely heavily on irrigation, changes in water availability are creating competition for resources, leading to conflict over water rights.

Water scarcity, in particular, has become a pressing issue in many arid and semi-arid regions, where rainfall is already scarce. As climate change exacerbates the issue, there is increasing concern over the future availability of freshwater supplies. This is particularly problematic in regions where large populations depend on already limited water resources for drinking, sanitation, and agriculture. Countries that are heavily reliant on glacial meltwater, such as those in the Himalayas or the Andes, face the additional challenge of diminishing glaciers, which provide an important source of freshwater during the dry season.

Public health is another area significantly impacted by climate change. Rising temperatures and the increased frequency of extreme weather events can exacerbate the spread of diseases such as malaria, dengue fever, and cholera. Warmer temperatures create favorable conditions for disease-carrying mosquitoes, while heavy rainfall and flooding can contaminate water supplies, increasing the risk of waterborne diseases. In many regions, particularly those in the Global South, public health systems are already struggling to cope with these new challenges.

Addressing the root causes and mitigating the effects of climate change will require a multi-disciplinary approach that integrates science, technology, and policy. Accurate predictive models are essential for forecasting regional changes in climate and developing strategies for adaptation and mitigation. These models will enable policymakers to anticipate future challenges and implement effective measures to reduce the risks associated with extreme weather events. For instance, better forecasting can help governments allocate resources more effectively, ensuring that communities are better prepared for floods, droughts, and storms.

Additionally, global cooperation in climate policy is essential to mitigate the effects of climate change. International agreements such as the Paris Agreement aim to limit global temperature rise to well below 2°C above pre-industrial levels, but much work remains to be done. Countries must continue to reduce their greenhouse gas emissions and transition to renewable energy sources to prevent the worst outcomes of climate change. At the same time, efforts must be made to promote resilience in vulnerable communities by investing in infrastructure, improving water management, and promoting sustainable agricultural practices.

In conclusion, the impact of climate change on regional weather patterns is far-reaching and poses significant challenges to both the natural world and human society. While adaptation and mitigation strategies are essential to minimize these impacts, it is equally important to take urgent action to reduce the root causes of climate change. Only through global cooperation, informed policy decisions, and sustainable practices can we ensure a livable future for generations to come.

%%%%%%%%%%%%%%%%%%%%%%FIgure 2.1%%%%%%%%%%%%%%%%%%%%%%%%%%%%%%%%%%%%%%%%%%%%%%%%%%%%%%%%%%
\begin{figure}
    \centering
    \includegraphics[width=1\linewidth]{Figures/C2/fig_2_1.jpg}
     \caption{Random meme from the internet to make this template more enjoyable.\\  
    \textbf{Inset}: The study region is marked with a \textcolor{red}{red} rectangle.}
    \label{fig: 1.1}
\end{figure}

%**************************************************************************
%**************************************************************************
\pagebreak
\begin{landscape} % Start landscape mode for the page
% \textbf{[Table 2.2]}
% Set temporary margins for the table page
\newgeometry{left=1.5cm, right=1.5cm, top=2cm, bottom=2cm}

\begin{table}[h]
\centering
\caption{List of recent (1960–2017) moderate and higher magnitude (\(M_c 5.5\)) earthquakes}
\label{tab:2.1}
\begin{tabular}{|c|c|c|c|c|c|c|c|}
\hline
Year & Month & Day & Magnitude (Mw) & Magnitude (Mb) & Magnitude (Ms) & Magnitude (Ml) \\ \hline
1960 & 2  & 5  & 5.6 & - & - & - \\ \hline
1961 & 6  & 27 & 5.5 & - & - & 6 \\ \hline
1962 & 5  & 4  & 5.6 & - & - & 5.6 \\ \hline
1963 & 11 & 24 & 5.5 & - & - & 5.8 \\ \hline
1964 & 7  & 10 & 5.7 & - & - & 5.7 \\ \hline
1965 & 10 & 2  & 5.5 & - & - & 5.6 \\ \hline
1966 & 8  & 13 & 5.8 & - & - & 5.7 \\ \hline
1967 & 9  & 23 & 5.7 & - & - & 5.6 \\ \hline
1968 & 1  & 7  & 5.8 & - & - & 5.9 \\ \hline
1969 & 3  & 3  & 5.7 & - & - & 6 \\ \hline
\end{tabular}
\end{table}

% Restore original margins after the table
\restoregeometry

\end{landscape}

%**************************************************************************

\subsection{Climate Data Compilation}

The climate data for the South Asian region is compiled using data from multiple authoritative sources spanning the period from 1800 to 2020. The synthesized dataset used in this study has been analyzed in both spatial and temporal domains to examine the trends and fluctuations in temperature and precipitation over time. The primary sources of input for this dataset are the Indian Meteorological Department (IMD), the World Meteorological Organization (WMO), the National Oceanic and Atmospheric Administration (NOAA), and the Climate Research Unit (CRU). Pre-twentieth-century records of temperature and rainfall have been sourced from \textcite{yadav_crustal_2022}. Additionally, historical climate data has been derived from earlier compilations by \textcite{babu_updated_2023}. From IMD, WMO, and NOAA, the data for the periods 1850-2000, 1901-2020, and 1950-2020 have been considered. The compilation process involved applying tolerance criteria to identify and eliminate duplicate entries based on the geographic coordinates (0.1\textdegree), temperature (1.0°C), and time (1 day). In the case of duplicate entries, the event from the dataset with the most recent records is prioritized. The priority order for the selection of data is IMD, NOAA, and WMO.\par

The spatial distribution of temperature anomalies and precipitation patterns over the study area is shown in Figure \ref{fig: 2.2}. The data includes temperature readings across all seasons, where most of the records pertain to summer and winter temperature extremes. Figure \ref{fig: 2.2} presents the distribution of temperature change versus the time of observation, highlighting the variation in temperature anomalies across different decades. With the evolution of climate monitoring systems, the precision and accuracy of temperature measurements have significantly improved over time, leading to a better understanding of long-term climate trends. The completeness of the climate dataset has greatly improved, especially in recent years due to advancements in satellite-based remote sensing and the expansion of the global weather monitoring network. Figure \ref{fig: 3.2} shows substantial improvements in data collection in the 1970s, 1990s, and 2010s, with more consistent and accurate records for all regions.\par

The climate dataset can be divided into three major periods. The period from 1800 to 1900 is considered as the historical phase for which reliable instrumental data is non-existent, but records of extreme events were officially documented (Table \ref{tab: 3.1}). During this time, there was considerable uncertainty regarding temperature values and rainfall patterns, and only a handful of major events, such as significant monsoons or droughts, were recorded. The early 20th century, from 1900 to 1950, provides more frequent records, though much of the data pertains to smaller-scale climatic events, with some reliable observations made using early meteorological instruments. The mid-century period also includes the extreme heatwave of 1947 that caused widespread agricultural damage across the region. In the post-1950 period, there was a significant improvement in climate monitoring systems, leading to an exponential increase in the number of recorded climatic events (Figure \ref{fig: 3.2}). The reliability of temperature and rainfall data has vastly improved due to the proliferation of weather stations and satellite systems.

