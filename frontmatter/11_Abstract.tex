\newpage

\begin{center}
{\LARGE \textbf{Abstract}}
\end{center}

% \section*{Abstract}
\addcontentsline{toc}{chapter}{Abstract}
\justifying
{\huge T}his thesis investigates advanced methodologies for seismic exploration, with a specific focus on enhancing subsurface imaging techniques in complex geological settings. By integrating traditional geophysical methods with modern computational approaches, this study aims to address challenges associated with the accurate interpretation of seismic data in regions with heterogeneous crustal structures.  

The research explores three key areas:  
1. Developing robust algorithms for seismic signal processing to improve the resolution of subsurface features.  
2. Applying machine learning techniques to predict lithological variations and fault geometries from seismic datasets.  
3. Conducting field experiments in diverse geological terrains, including sedimentary basins and active tectonic regions, to validate the proposed methodologies.  

Findings from this research indicate a significant improvement in the accuracy of velocity models and fault delineation, particularly in regions characterized by complex stratigraphy. The integration of artificial intelligence in seismic data interpretation has further enhanced the prediction of subsurface properties, offering a transformative approach for both academic research and industrial applications.  

This work has implications for natural resource exploration, hazard assessment, and understanding the geodynamic processes shaping the Earth's crust. Future directions include refining the algorithms for real-time data processing and extending the methodologies to offshore exploration environments.  

\textbf{Keywords:} Seismic Exploration, Subsurface Imaging, Machine Learning, Geophysics  



